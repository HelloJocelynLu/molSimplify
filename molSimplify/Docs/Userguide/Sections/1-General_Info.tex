\section{General information}

molSimplify is an open source utility that incorporates geometric manipulation routines necessary for the generation of transition metal complexes, automated setup and completion of electronic structure calculations, post-processing and data analysis. The software generates a variety of coordination complexes with any number of metals coordinated by ligands in a single or multidentate (chelating) fashion. The code can both build the coordination complex starting from a single metal atom or work to functionalize a more complex structure (e.g. a porphyrin or other metal-ligand complex) by including additional ligands or replacing existing ones. molSimplify builds intermolecular complexes for evaluating binding interactions and generating candidate reactants and intermediates for catalyst reaction mechanism screening and also supports interaction with chemical databases. Furthermore, it provides a Graphical User Interface (GUI) and is thus accessible to a wider audience since it does not require a lot of prior computational chemistry experience.

\subsection{Obtaining molSimplify}
The binaries, source code and useful documentation can be obtained online by visiting: \url{http://molsimplify.mit.edu}. 


\subsection{Citing molSimplify}

Any published work that utilizes molSimplify shall include the following reference:\\ 

\bibentry{Ioannidis2016}
\newpage{}
\subsection{License}
\subsubsection{molSimplify}
The software is distributed free of charge under the GPL license: \\  \\
\textit{Copyright 2016 Kulik Lab @ MIT\\ \\
    molSimplify is free software: you can redistribute it and/or modify it under the terms of the GNU General Public License as published by the Free Software Foundation, either version 3 of the License, or (at your option) any later version. molSimplify is distributed in the hope that it will be useful,  but WITHOUT ANY WARRANTY; without even the implied warranty of MERCHANTABILITY or FITNESS FOR A PARTICULAR PURPOSE.\\    See the GNU General Public License for more details.     You should have received a copy of the GNU General Public License 
    along with molSimplify. If not, see http://www.gnu.org/licenses/.
}
\subsubsection{Pybrain}
molSimplify makes use of the open-source Pybrain library (\url{http://pybrain.org/pages/home}) for neural network implementation. This work is not endorsed by Pybrain. Pybrain is licensed under the BSD licence:\\

\textit{Copyright 2008, IDSIA and TU M\"{u}nchen (TUM)
All rights reserved.}
\\
\textit{Redistribution and use in source and binary forms, with or without
modification, are permitted provided that the following conditions are met:}
\begin{itemize}
   \item \textit{Redistributions of source code must retain the above copyright
      notice, this list of conditions and the following disclaimer.}
    \item \textit{Redistributions in binary form must reproduce the above copyright
      notice, this list of conditions and the following disclaimer in the
      documentation and/or other materials provided with the distribution.}
    \item \textit{Neither the name of the associated institutions nor the
      names of its contributors may be used to endorse or promote products
      derived from this software without specific prior written permission.}
\end{itemize}

\textit{THIS SOFTWARE IS PROVIDED BY THE DEVELOPERS ''AS IS'' AND ANY
EXPRESS OR IMPLIED WARRANTIES, INCLUDING, BUT NOT LIMITED TO, THE IMPLIED
WARRANTIES OF MERCHANTABILITY AND FITNESS FOR A PARTICULAR PURPOSE ARE
DISCLAIMED. IN NO EVENT SHALL THE DEVELOPERS OR THEIR INSTITUTIONS BE LIABLE FOR ANY
DIRECT, INDIRECT, INCIDENTAL, SPECIAL, EXEMPLARY, OR CONSEQUENTIAL DAMAGES
(INCLUDING, BUT NOT LIMITED TO, PROCUREMENT OF SUBSTITUTE GOODS OR SERVICES;
LOSS OF USE, DATA, OR PROFITS; OR BUSINESS INTERRUPTION) HOWEVER CAUSED AND
ON ANY THEORY OF LIABILITY, WHETHER IN CONTRACT, STRICT LIABILITY, OR TORT
(INCLUDING NEGLIGENCE OR OTHERWISE) ARISING IN ANY WAY OUT OF THE USE OF THIS
SOFTWARE, EVEN IF ADVISED OF THE POSSIBILITY OF SUCH DAMAGE}


\subsection{Acknowledgments}
This software was developed by Efthymios I. Ioannidis, JP Janet, Terry Z. H. Gani  and Heather J. Kulik at the Massachusetts Institute of Technology. The authors would like to especially thank Terry Z. H. Gani for thoroughly testing the code and providing valuable feedback.