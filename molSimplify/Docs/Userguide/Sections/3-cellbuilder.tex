\section{Cell Builder Module}

molSimplify includes tool form placing generated molecules on surfaces in order to study supported complexes and surface-adsorbate interactions, as well as some tools for creating supercells/slabs from basic input data. 

\subsection{Slab generation control}
The basic requirements for generating 
The slab construction sub-module has the ability to perform this rotation and cutting along user-specified Miller indices for cubic crystal types.





\subsection{Slab size control}
Once a slab has been generated, the second sub-module allows the placement of molecules, complexes or  structures on the generated supercell surface. Any object that can be generated by molSimplify can be placed on the surface, and the object is specified using the molSimplify syntax. The code supports a variety of options for controlling the surface placement, which allows for automated investigation of various scenarios of interest for computational studies of catalysis and surface chemistry. Combining the placement options with the toolbox provided by molSimplify allows for placement of extra molecules to investigate non-covalent interactions. 

The placement algorithm supports the followings options: centered, staggered, and atom-to-atom. In all cases, the user must specify how far the placed object will be from the surface. If physisorption is specified, the van der Waals radii are used, and if chemisorption is specified, the covalent radii are used. The user can also overwrite the distance to a custom value.  In the centered method, the adsorbed object is placed such that the center of mass of the object is directly above the symmetric center of the supercell surface.  In the staggered mode, it is placed directly above the midpoint between surface sites closest to the center of symmetry of the surface. 

The most flexible placement method is to align-by-atom, where the user supplies a type of surface atom (for example, O in TiO2), and a list of either atom types or indices (one-indexed, in the order of geometry file) for the adsorbed object. The specified object will be attached to target sites such that the distance between each of the selected atoms in the object is as close as possible to the target distance, as measured by the sum of square error between the resulting distances and the specified targets.

In a similar fashion to the manner in which the primary molSimplify code places ligands based on denticitiy, the process that is followed depends on the number of atoms in the object that are selected for alignment. First, a matching number of target sites on the surface are selected – the user can determine if initial choice should be uniform random or centred in the unit cell. If a single alignment atom is selected, then the code will align the object such that the selected atom is directly above the target site at the specified distance, and then attempt to rotate the attached object around the alignment axis in order to maximize the interatomic distance between the attached object and the cell (or other, previously attached objects).

For two or more requested atoms, it is not possible in general to satisfy the requested distance exactly. Instead, the attached group is lowered towards the surface incrementally, and at each step, rotated around the axis joining the unweighted center of mass of the chosen atoms and the target surface atoms, to minimize the sum-of-squared error between the current distances and the requested value. The step is only accepted if the error decreases – if the step fails, it is retried with the approach distance halved. The process terminates when two sequential steps fail.

Depending on the number of atoms and their relative positions, the initial alignment above the surface can include additional rotations, depending on the symmetry of the selected atoms. If two atoms are selected, or all of the atoms are collinear, the object can be rotated around this axis to maximize intermolecular spacing, and in the case co-planar atoms, the entire object can be rotated to align the plane with the surface.
 
The placement algorithm also supports multiple adsorbates, either by specifying a number of objects to place, or by requesting a coverage fraction. In either case, the surface sites that are selected are removed from the list of the available sites, and the attachment process is repeated the required number of times. In order to select the next attachment site, the algorithm combinatorically tests all remaining sites and selects the one that maximizes the minimum inter- adsorbate spacing. The distance metric must respect the periodicity of the unit cell by testing the distance to the adjacent repeat cells in x-y plane (for cubic geometries, this is the Euclidean distance on 2D torus). 
